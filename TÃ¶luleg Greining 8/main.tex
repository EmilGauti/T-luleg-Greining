\documentclass[11pt]{article}
\usepackage{amsmath,amssymb, amsthm, marvosym, permute, extsizes}
\usepackage{siunitx, graphicx, float, enumitem, adjustbox, hyperref, bm}
\usepackage{microtype, dsfont}
\usepackage[normalem]{ulem}
\usepackage[T1]{fontenc}
\usepackage[utf8]{inputenc}
\usepackage{lmodern}
\usepackage[T1]{fontenc}
\usepackage[a4paper,margin=2.5cm]{geometry}
\usepackage[icelandic]{babel}
\newcommand{\explain}[2]{\underbrace{#1}_\textrm{$#2$}}

\usepackage{minted}

\title{Heimadæmi 8\\ \vspace{0.4cm} \large Töluleg Greining}
\author{Emil Gauti Friðriksson}
\begin{document}
\maketitle
\section*{dæmi 3 í Computer Problems 4.3}
implement Householder reflections in:\\
Write a Matlab program that implements classical Gram–Schmidt to find the reduced QR
factorization. Check your work by comparing factorizations of the matrices in Exercise 1 with
the Matlab qr(A,0) command or equivalent. The factorization is unique up to signs of the
entries of Q and R.
\subsection*{Svar}

\begin{minted}{matlab}
function [Q,R] = verkefni81(A) 
[m,n] = size(A);
R = A;
Q = eye(m);
H = cell(1,m-1);
H{1} = eye(m);

    for i = 1:m-1
        I = eye(m);
        x0 = H{i}*A;
        x = x0(i:m,i);
        w = [norm(x); zeros(m-i,1)];
        v = w - x;
        p = v*v'/(v'*v);
        h = eye(m-(i-1)) - 2*p;
        I(i:m,i:m) = h;
        H{i+1} = I;
    end
    
    for i = 1:m-1
       R = H{i+1}*R;
    end
    
    for i = m-1:-1:1
       Q = H{i+1}*Q;
    end
    A

end
\end{minted}
ef notað á fylkin úr exercise 1 úr kaflanum
fæst eftirfarandi:
\begin{minted}{console}
verkefni81(A)
A =

     4     0
     3     1
ans =
    0.8000    0.6000
    0.6000   -0.8000
    
    
verkefni81(B)
A =
     1     2
     1     1
ans =
    0.7071    0.7071
    0.7071   -0.7071

verkefni81(C)
A =
     2     1
     1    -1
     2     1
ans =
    0.6667    0.2357    0.7071
    0.3333   -0.9428    0.0000
    0.6667    0.2357   -0.7071
    
verkefni81(D)
A =
     4     8     1
     0     2    -2
     3     6     7
ans =
    0.8000         0   -0.6000
         0    1.0000         0
    0.6000         0    0.8000   
\end{minted}
Þetta er allt í samræmi við svörin.


\end{document}
