\documentclass[11pt]{article}
\usepackage{amsmath,amssymb, amsthm, marvosym, permute, extsizes}
\usepackage{siunitx, graphicx, float, enumitem, adjustbox, hyperref, bm}
\usepackage{microtype, dsfont}
\usepackage[normalem]{ulem}
\usepackage[T1]{fontenc}
\usepackage[utf8]{inputenc}
\usepackage{lmodern}
\usepackage[T1]{fontenc}
\usepackage[a4paper,margin=2.5cm]{geometry}
\usepackage[icelandic]{babel}

\usepackage{minted}

\title{Heimadæmi 2\\ \vspace{0.4cm} \large Töluleg Greining}
\author{Emil Gauti Friðriksson}
\begin{document}
\maketitle
\section*{Dæmi 1}
\subsection*{1.1 Computer Problems, dæmi 7}
Use the Bisection Method to find the two real numbers $x$, within six correct decimal places,
that make the determinant of the matrix
$$A =
\begin{bmatrix}
1 & 2  & 3  & x\\
4 & 5  & x  & 6\\
7 & x  & 8  & 9\\
x & 10 & 11 & 12
\end{bmatrix}
$$
equal to 1000. For each solution you find, test it by computing the corresponding determinant
and reporting how many correct decimal places (after the decimal point) the determinant has
when your solution x is used. You may use the Matlab command det to compute the
determinants.
\section*{Svar}
Notast er við \mintinline{matlab}|bisect.m| forritið úr bókinni, fundin voru tvö bil sem innuhéldu eina rót fyrir sig, aðferðin við að finna þessi bil var fremur frumstæð en notast var við ágiskun.
\begin{minted}{matlab}
format long
syms x;
A = [1, 2, 3, x; 4, 5, x, 6; 7, x, 8, 9; x, 10, 11, 12 ];
f(x) = det(A)-1000;
a = [-20, 0]; 
b = [0, 20]; 
for i=1:2
    bisect(f,a(i),b(i),0.000000005)
end
\end{minted}
fyrra bilið er [-20,0] og það seinna er [0,20], Þetta skilar eftirfarandi:
\begin{minted}{matlab}
ans = -17.188498149625957

ans =   9.708299119956791
\end{minted}
Þar sem \mintinline{matlab}|TOL = 0.000000005| þá getum við verið sannfærð um að svarið er a.m.k. nákvæmt upp að sex aukastöfum þ.e. $x_1 = -17.188498$ og $x_2 = 9.708299$

\noindent Látum nú $x=x_1$ og reiknum $\det(A(x_1))-1000 = -0.00180775$\\
Látum nú $x=x_2$ og reiknum $\det(A(x_2))-1000 = -1.40747 \cdot 10^{-4}$\\
Sjáum að raunverulegar rætur fallsins eru því ekki fjarri.

\section*{Dæmi 2}
\subsection*{1.2 Exercises, dæmi 14}
Which of the following three Fixed-Point Iterations converge to $\sqrt[]{2}$? Rank the ones that converge from fastest to slowest.\\
$$(A)\quad x\rightarrow \frac 12 x+\frac 1x \qquad (B)\quad x\rightarrow \frac 23 x+\frac{2}{3x} \qquad (C)\quad x\rightarrow \frac 34 x + \frac{1}{2x} $$
\subsection*{Svar}
Athugum hvert tilfelli með fallinu \mintinline{matlab}|fpi.m| sem gefið er í bókinni. Tökum 12 ítranir. skráum niður gildin og berum þau saman. Við drögum $\sqrt[]{2}$ frá því við höfum bara áhuga á hvort/hve hratt föllin nálgist $\sqrt[]{2}$.
\begin{minted}{matlab}
a=@(x) (1/2)*x+(1/x);
b=@(x) (2/3)*x+2/(3*x);
c=@(x) (3/4)*x+1/(2*x);

for i=1:12
   gildi_a(i) = fpi(a,1,i)-sqrt(2);
   gildi_b(i) = fpi(b,1,i)-sqrt(2);
   gildi_c(i) = fpi(c,1,i)-sqrt(2);
end
\end{minted}
Setjum gögnin fram í töflu, námundum upp að sjötta aukastaf:
\begin{table}[h]
\centering
\begin{tabular}{ l l l l}
\hline
i 		& $\quad(A)$ & $\quad(B)$ & $\quad(C)$\\
\hline
1  	& 0.085786 	& -0.080880 & -0.164214\\
2 	& 0.002453	& -0.025324 & -0.076714\\
3	& 0.000002 	& -0.008287 & -0.037257\\
4	& 0.000000	& -0.002746 & -0.018376\\
5	& 0.000000	& -0.000914 & -0.009128\\
6	& 0.000000	& -0.000304 & -0.004549\\
7	& 0.000000	& -0.000101 & -0.002271\\
8	& 0.000000	& -0.000034 & -0.001135\\
9	& 0.000000	& -0.000011 & -0.000567\\
10	& 0.000000	& -0.000004 & -0.000283\\
11	& 0.000000	& -0.000001 & -0.000142\\
12	& 0.000000	& 0.000000 & -0.000071\\
\hline
\end{tabular}
\end{table}\\
Nú sjáum við að jafna (A) er hröðust að nálgast $\sqrt[]{2}$, þar á eftir kemur (B) og loks (C), öll tilfellin nálgast $\sqrt[]{2}$.

\newpage
\section*{Dæmi 3}
\subsection*{1.5 Exercises}

\subsection*{Dæmi 1.5.1}

Apply two steps of the Secant Method to $e^x + \sin x = 4$ with initial guesses $x_0=1$ and $x_1=2.$
\subsection*{Svar}


\begin{align*}
x_{i+1} = x_i - \frac{f(x_i)(x_i-x_{i-1})}{f(x_i)-f(x_{i-1})}
\end{align*}
svo við fáum eftirfarandi:
\begin{align*}
x_2 &= x_1 - \frac{f(x_1)(x_1-x_0)}{f(x_1)-f(x_0)}\\
	&= 2 - \frac{(e^2+\sin(2)-4)(2-1)}{(e^2+\sin(2)-4)-(e^1+\sin(1)-4)}\\
    &\approx 1.0929
\end{align*}
Höldum áfram og reiknum $x_3$
\begin{align*}
x_3 &= 1.0929 - \frac{(e^{1.0929}+\sin(1.0929)-4)(1.0929-2)}{(e^{1.0929}+\sin(1.0929)-4)-(e^2+\sin(2)-4)}\\
	&\approx \underline{1.1194}
\end{align*}




\subsection*{Dæmi 1.5.2}
Apply two steps of the Method of False Position with initial bracket [1, 2] to the equations of
Exercise 1.
\subsection*{Svar}
skrifum $f(x) = e^x + \sin x -4$
\begin{align*}
x_2 &= \frac{x_1f(x_0)-x_0f(x_1)}{f(x_0)-f(x_1)}\\
	&= \frac{2(e+\sin(1)-4)-1(e^2+\sin(2)-4)}{(e+\sin(1)-4)-(e^2+\sin(2)-4)}\\
    &\approx 1.0929
\end{align*}
athugum síðan að $f(x_2)\cdot f(x_1) < 0$ því er rótin á bilinu $[x_1, x_2]$. Beitum því sömu útreikningum til að finna $x_3$.
\begin{align*}
x_3 &= \frac{x_1f(x_2)-x_2f(x_1)}{f(x_2)-f(x_1)}\\
	&= \frac{2(e^{1.0929}+\sin(1.0929)-4)-1.0929(e^2+\sin(2)-4)}{(e^{1.0929}+\sin(1.0929)-4)-(e^2+\sin(2)-4)}\\
	&\approx \underline{1.1194}
\end{align*}

\newpage
\subsection*{Dæmi 1.5.3}
Apply two steps of Inverse Quadratic Interpolation to the equations of Exercise 1. Use initial
guesses $x_0 = 1$, $x_1=2$, and $x_2 = 0$, and update by retaining the three most recent iterates.
\subsection*{Svar}
látum $a = x_0$, $b=x_1$ og $c=x_2$\\
Þá höfum við eftirfarandi: \\
$f(a) = A = e+\sin(1)-4$, $f(b) = B = e^2 +\sin(2) -4$ og $f(c) = C = -3$\\
skilgreinum síðan eftirfarandi:
$q=f(a)/f(b) \approx -0.1024$, $r=f(c)/f(b)\approx -0.6979$ og $s=f(c)/f(a)\approx -25.9752$\\
Inverse Quadratic Interpolation gefur síðan:
\begin{align*}
x_{i+3} = x_{i+2}- \frac{r(r-q)(x_{i+2}-x_{i+1})+(1-r)s(x_{i+2}-x_i)}{(q-1)(r-1)(s-1)}
\end{align*}
og því höfum við að $x_3$ er eftirfarandi:
\begin{align*}
x_3 &= 0 - \frac{-0.6979(-0.6979 - (-0.1024))(0-1) + (1-(-0.1024))(-25.9752)(0-1)}{(-0.1024-1) (-0.6979-1) (-25.9752-1)} \\
	&\approx 0.5589
\end{align*}
nú er $a=2$, $b=0$ og $c=0.5589$\\
$q = f(a)/f(b) = -1.4328$, $r=f(c)/f(b)\approx 0.5737$ og $s=f(c)/f(a)\approx -0.4004$\\
Nú getum við því fundið $x_4$:
\begin{align*}
x_4 &= 0.5589 - \frac{0.5737(0.5737-(-1.4328))(0.5589-0)+(1-0.5737)(-0.4004)(0.5589-2)}{(-1.4328-1) (0.5737-1) (-0.4004-1)}\\
	&\approx \underline{1.1712} 
\end{align*}








\end{document}
