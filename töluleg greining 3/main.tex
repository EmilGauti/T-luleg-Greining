\documentclass[11pt]{article}
\usepackage{amsmath,amssymb, amsthm, marvosym, permute, extsizes}
\usepackage{siunitx, graphicx, float, enumitem, adjustbox, hyperref, bm}
\usepackage{microtype, dsfont}
\usepackage[normalem]{ulem}
\usepackage[T1]{fontenc}
\usepackage[utf8]{inputenc}
\usepackage{lmodern}
\usepackage[T1]{fontenc}
\usepackage[a4paper,margin=2.5cm]{geometry}
\usepackage[icelandic]{babel}

\usepackage{minted}

\title{Heimadæmi 3\\ \vspace{0.4cm} \large Töluleg Greining}
\author{Emil Gauti Friðriksson}
\begin{document}
\maketitle
\section*{Dæmi 1}
\subsection*{2.4 Exercises, dæmi 4}
Solve the system by finding the PA=LU factorization and then carrying out the two-step back
substitution.

\begin{align*}
(a)\quad
\begin{bmatrix}
4 & 2 & 0\\
4 & 4 & 2\\
2 & 2 & 3
\end{bmatrix}
\begin{bmatrix}
x_1\\
x_2\\
x_3
\end{bmatrix}
=
\begin{bmatrix}
2\\
4\\
6
\end{bmatrix}
\quad (b) \quad
\begin{bmatrix}
-1 & 0 & 1\\
2  & 1 & 1\\
1  & 1 & 0
\end{bmatrix}
\begin{bmatrix}
x_1\\
x_2\\
x_3
\end{bmatrix}
=
\begin{bmatrix}
-2\\
17\\
3
\end{bmatrix}
\end{align*}
\subsection*{Svar(a)}
(1) Skiptum á línu 1 og línu 2\\
(2) drögum línu 1 frá línu 2 og drögum $\frac 12$ línu 1 frá línu 3\\
Höfum þá PA = LU:
\begin{align*}
\begin{bmatrix} % P
0 & 1 & 0\\
1 & 0 & 0\\
0 & 0 & 1
\end{bmatrix}
\begin{bmatrix} % A
4 & 2 & 6\\
4 & 4 & 2\\
2 & 2 & 3
\end{bmatrix}
=
\begin{bmatrix} % L
1 & 0 & 0\\
1 & 1 & 0\\
1/2 & 0 & 1
\end{bmatrix}
\begin{bmatrix} % U
4 & 4 & 2\\
0 & -2& -2\\
0 & 0 & 2
\end{bmatrix}
\end{align*}
Leysum síðan Lc = Pb fyrir c.
\begin{align*}
\begin{bmatrix} % L
1 & 0 & 0\\
1 & 1 & 0\\
1/2 & 0 & 1
\end{bmatrix}
\begin{bmatrix}
c_1\\
c_2\\
c_3
\end{bmatrix}
=
\begin{bmatrix} % P
0 & 1 & 0\\
1 & 0 & 0\\
0 & 0 & 1
\end{bmatrix}
\begin{bmatrix}
2\\
4\\
6
\end{bmatrix}
\end{align*}
sem gefur síðan
\begin{align*}
c_1 &= 4\\
c_2 = 2 - c_1 &= -2\\
c_3 = 6 - \frac 12 c_1 &= 4
\end{align*}
Leysum nú Ux=c fyrir x.
\begin{align*}
\begin{bmatrix} % U
4 & 4 & 2\\
0 & -2& -2\\
0 & 0 & 2
\end{bmatrix}
\begin{bmatrix}
x_1\\
x_2\\
x_3
\end{bmatrix}
=
\begin{bmatrix}
4\\
-2\\
4
\end{bmatrix}
\end{align*}
sem gefur okkur
\begin{align*}
x_3 = \frac 42 &= 2\\
x_2 = \frac{-2 +2 x_3}{-2} &= -1\\
x_1 = \frac{4-2x_3-4x_2}{4} &= 1
\end{align*}
svo við fáum að $[x_1, x_2, x_3] = [1, -1, 2]$


\subsection*{Svar(b)}
\textbf{ATH!} Ég setti óvart '1' í stað '2' í línu 3 dálk 2, ég fæ því annað svar, ég vona að það sé afsakanlegt :) \\ 
\medskip

\noindent(1) skiptum á línu 2 og línu 1\\
(2) skiptum á línu 2 og línu 3\\
(3) drögum $\frac 12$ línu 1 frá línu 2\\
(4) drögum $-\frac 12$ línu 1 frá línu 3\\
(5) drögum línu 2 frá línu 3\\
Höfum þá PA = LU
\begin{align*}
\begin{bmatrix} % P
0 & 1 & 0\\
0 & 0 & 1\\
1 & 0 & 0
\end{bmatrix}
\begin{bmatrix} % A
-1 & 0 & 1\\
2  & 1 & 1\\
1  & 1 & 0
\end{bmatrix}
=
\begin{bmatrix} % L
1 & 0 & 0\\
1/2 & 1 & 0\\
-1/2 & 1 & 1
\end{bmatrix}
\begin{bmatrix} % U
2 & 1 & 1\\
0 & 1/2 & -1/2\\
0 & 0 & 2
\end{bmatrix}
\end{align*}
Leysum síðan Lc = Pb fyrir c.
\begin{align*}
\begin{bmatrix} % L
1 & 0 & 0\\
1/2 & 1 & 0\\
-1/2 & 1 & 1
\end{bmatrix}
\begin{bmatrix}
c_1\\
c_2\\
c_3
\end{bmatrix}
=
\begin{bmatrix} % P
0 & 1 & 0\\
0 & 0 & 1\\
1 & 0 & 0
\end{bmatrix}
\begin{bmatrix}
-2\\
17\\
3
\end{bmatrix}
=
\begin{bmatrix}
17\\
3\\
-2
\end{bmatrix}
\end{align*}
fáum úr þessu:
\begin{align*}
c_1 &= 17\\
c_2 = 3 - \frac 12 c_1 &= -\frac{11}{2}\\
c_3 = -2 - c_2 + \frac 12 c_1 &= 12
\end{align*}
Leysum nú Ux=c fyrir x.
\begin{align*}
\begin{bmatrix} % U
2 & 1 & 1\\
0 & 1/2 & -1/2\\
0 & 0 & 2
\end{bmatrix}
\begin{bmatrix}
x_1\\
x_2\\
x_3
\end{bmatrix}
=
\begin{bmatrix}
17\\
-11/2\\
12
\end{bmatrix}
\end{align*}
Fáum úr þessu:
\begin{align*}
x_3 &= 6\\
x_2 = 2(-11/2 + \frac{1}{2} x_3) &= -5\\
x_1 = \frac{17-x_3-x_2}{2} &= 8
\end{align*}
Fáum því að lokum að $[x_1, x_2, x_3] = [8, -5 , 6]$

\section*{Dæmi 2}
\subsection*{Computer problems 2.2, dæmi 1 og 2}

\textbf{Dæmi 1} Use the code fragments for Gaussian elimination in the previous section to write a Matlab
script to take a matrix A as input and output L and U. No row exchanges are allowed\\
the program should be designed to shut down if it encounters a zero pivot. Check your program by factoring the matrices in Exercise 2.\\

\noindent \textbf{Dæmi 2} Add two-step back substitution to your script from Computer Problem 1, and use it to solve the
systems in Exercise 4.

\begin{minted}{matlab}
function [L, U, x] = naiveGauss(A, b)
s = size(A);
if s(1) ~= s(2); error('Not nxn matrix');
end
n = s(1);
L = zeros(s(1), s(2));
U = L;
for j = 1 : n-1
if abs(A(j,j))<eps; error('Zero pivot encountered'); 
end
for i = j+1 : n
 mult = A(i,j)/A(j,j);
L(i,j) = mult;
for k = j+1 : n
A(i,k) = A(i,k) - mult*A(j,k);
end
end
end
for i = 1:n
L(i,i) = 1;
end
for i = 1:n
for j = i:n
U(i,j) = A(i,j);
end
end
c = zeros(n, 1);
for i = 1 : n
for j = 1 : i
b(i) = b(i) - L(i,j)*c(j);
end
c(i) = b(i)/L(i,i);
end
x = zeros(n, 1);
for i = n : -1 : 1
for j = i+1 : n
c(i) = c(i) - U(i,j)*x(j);
end
x(i) = c(i)/U(i,i);
end
end
\end{minted}
og ef sett eru fylkin úr 2.4 exercises fáum við
\begin{minted}{matlab}
A1 = [3 1 2; 6 3 4; 3 1 5]
b1 = [0 1 3]
A2 = [4 2 0; 4 4 2; 2 2 3]
b2 = [2 4 6]
[L1, U1, x1] = naiveGauss(A1, b1)
[L2, U2, x2] = naiveGauss(A2, b2)

L1 =
     1     0     0
     2     1     0
     1     0     1
U1 =
     3     1     2
     0     1     0
     0     0     3
x1 =
    -1
     1
     1
L2 =
    1.0000         0         0
    1.0000    1.0000         0
    0.5000    0.5000    1.0000
U2 =
     4     2     0
     0     2     2
     0     0     2
x2 =
     1
    -1
     2
\end{minted}



















\end{document}
