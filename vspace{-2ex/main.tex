\documentclass[11pt]{article}
\usepackage{amsmath,amssymb, amsthm, marvosym, permute, extsizes}
\usepackage{siunitx, graphicx, float, enumitem, adjustbox, hyperref, bm}
\usepackage{microtype, dsfont}
\usepackage[normalem]{ulem}
\usepackage[T1]{fontenc}
\usepackage[utf8]{inputenc}
\usepackage{lmodern}
\usepackage[T1]{fontenc}
\usepackage[a4paper,margin=2.5cm]{geometry}
\usepackage[icelandic]{babel}
\usepackage{tikz}
\newcommand{\explain}[2]{\underbrace{#1}_\textrm{$#2$}}

\usepackage{minted}

\title{\vspace{-2ex}Heimadæmi 11\\ \vspace{0.2cm} \large Töluleg Greining\vspace{-2ex}}
\author{Emil Gauti Friðriksson}
\begin{document}
\maketitle
\section*{Dæmi 1(i)}
\fbox{
\begin{minipage}{.5\textwidth}
\textbf{(a)} ætlum að leysa verkefnið
\begin{align*}
y' &= 2(t+1)y\\
y(0) &= 1
\end{align*}
höfum því eftirfarandi:
\begin{align*}
\frac{dy}{dt} &= 2ty + 2y\\
\frac{1}{y} dy &= 2t +2 dt\\
\ln(y)	&= t^2+2t\\
y		&=e^{t^2+2t} + C\\
\end{align*}
\noindent Upphafsskilyrðið gefur okkur síðan að $C=0$ svo við höfum að
\begin{align*}
y(t) = e^{t^2+2t}
\end{align*}
\end{minipage}
}
\fbox{
\begin{minipage}{.5\textwidth}
\noindent \textbf{(b)} Við ætlum að leysa verkefnið
\begin{align*}
y' = t^3/y^2\\
y(0)=1
\end{align*}
höfum því eftirfarandi
\begin{align*}
\frac{dy}{dt} &= \frac{t^3}{y^2}\\
y^2dy &= t^3dt\\
\frac 13 y^3 &= \frac 14 t^4\\
y &= \sqrt[3]{3/4 t^4 +C}\\
\end{align*}
Upphafsskilyrðið gefur okkur síðan að C=1 svo við höfum að
\begin{align*}
y(t) = \sqrt[3]{\frac{3}{4} t^4 +1}
\end{align*}
\end{minipage}
}
\section*{Dæmi 1(ii)}
\textbf{(a)}\\
Nú ætlum við að meta villuna við nálgunarlausnina þegar við notum forritið \mintinline{matlab}{euler.m} úr bók(bls. 286), Ath! við breytum neðstu línunni \mintinline{matlab}{z=2*t*y+2*y;} í fyrrnefndi forriti. Keyrum síðan eftirfarandi skipun:
\begin{minted}{matlab}
[t, nalgun] = euler([0 1],1,4);
rett = exp(t.^2).*exp(2.*t);
nalgun
villa = abs(nalgun - rett)
\end{minted}
og fáum út eftirfarandi:
\begin{minted}{matlab}
nalgun =
    1.0000    1.5000    2.4375    4.2656    7.9980
villa =
         0    0.2551    1.0528    3.6000   12.0875
\end{minted}

\noindent \textbf{(b)}\\
Sambærilegur við liðinn hérna á undan nema núna breytum við neðstu línunni í forritinu í \mintinline{matlab}{z=t^3/y^2}. Við breytum einnig skipununm í eftirfarandi:
\begin{minted}{matlab}
[t, nalgun] = euler([0 1],1,4);
rett = (3/4.*t.^4+1).^(1/3);
nalgun
villa = abs(nalgun - rett)
\end{minted}
\noindent Fáum eftirfarandi niðurstöður:
\begin{minted}{matlab}
nalgun =
    1.0000    1.0000    1.0039    1.0349    1.1334
villa =
         0    0.0010    0.0115    0.0386    0.0717

\end{minted}

\section*{Dæmi 1(iii)}
\textbf{(a)}\\
hérna ætlum við að nýta okkur forritið \mintinline{matlab}{predcorr.m}(bls. 342) úr bókinni, forritið notar trapízuheildun en við nýtum okkur reyndar bara hluta af forritinu \\
Við þurfum að breyta síðustu línum forritsins í eftirfarandi:
\begin{minted}{matlab}
function z=ydot(t,y) % IVP
z=2*t*y+2*y;
\end{minted}
og þegar við keyrum forritið fáum við:
\begin{minted}{matlab}
>> [t,y]=predcorr([0 1],1,4,5)
t =
         0    0.2500    0.5000    0.7500    1.0000
y =
    1.0000
    1.7188
    3.3032
    7.0710
   16.7935
\end{minted}
sem hefur svo villuna í endapunktinum:
\begin{minted}{matlab}
abs(16.7935 - exp(3)) = 3.2920
\end{minted}
\textbf{(b)}\\
gerum sambærilega útreikninga fyrir seinni liðinn, núna verður síðasti hluti forritsins eftirfarandi:
\begin{minted}{matlab}
function z=ydot(t,y) % IVP
z=t^3/y^2;
\end{minted}
sem skilar svo þegar það er keyrt:
\begin{minted}{matlab}
>> [t,y]=predcorr([0 1],1,4,5)

t =
         0    0.2500    0.5000    0.7500    1.0000
y =
    1.0000
    1.0020
    1.0193
    1.0823
    1.2182
\end{minted}
sem hefur svo villuna í endapunktinum:
\begin{minted}{matlab}
abs(1.2182 - 1.75^(1/3)) = 0.0131
\end{minted}












\section*{Dæmi 2(i)}
Ætlum að breyta eftirfarandi diffurjöfnu í diffurjöfnu-hneppi af lægra stigi:
\begin{align*}
y''-2ty'+2y=0
\end{align*}
nú umskrifum við þetta á viðráðanlegra form:
\begin{align*}
y'' = 2ty'-2y
\end{align*}
Skilgreinum svo $y_1 = y$ og $y_2=y'$ þá getum við ritað:
\begin{align*}
y_2 &= y_1'\\
y_2'&= 2ty_2-2y_1
\end{align*}

\section*{Dæmi 2(ii)}
Nú ætlum við að nýta okkur forritið \mintinline{matlab}{pend.m}(bls. 307), sem nýtir sér trapísuheildun, við höfum áhuga á vigrinum $y$ sem við látum forritið skila okkur. við þurfum samt að lagfæra neðstu línurnar svo við fáum réttar niðurstöður þær verða eftirfarandi:
\begin{minted}{matlab}
function z=ydot(t,y)
z(1)=y(2);
z(2)=2*t*z(1)-2*y(1);
\end{minted}
og þegar við keyrum þetta með upphafsgildunum okkar og skoðum bilið [0, 1] fáum við eftirfarandi:
\begin{minted}{matlab}
>> y=pend([0 1],[1 1],4)
\end{minted}
\begin{minted}{matlab}
y =
    1.0000    1.0000
    1.1875    0.4688
    1.2378   -0.1333
    1.1229   -0.9078
    0.7832   -2.0352
\end{minted}
Hér er vinstri dálkurinn $y_1 = y$ og hægri dálkurinn $y_2=y'$
\end{document}
